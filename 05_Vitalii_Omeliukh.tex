\documentclass[a4paper,12pt]{article}
\usepackage[a4paper, margin=0.9in]{geometry} % Встановлюємо вузькі поля
\usepackage[utf8]{inputenc}
\usepackage[T2A]{fontenc}
\usepackage[polish]{babel}
\usepackage{titlesec}
\usepackage{tocloft}
\usepackage{hyperref}
\usepackage{graphicx}
\DeclareGraphicsExtensions{.pdf,.png,.jpg}
\hypersetup{
	colorlinks=true,
	linkcolor=blue,
	urlcolor=blue,
}

\titleformat{\section}[block]{\bfseries\Large}{}{0em}{}
\titleformat{\subsection}[block]{\bfseries\normalsize}{}{0em}{}

\renewcommand{\cftsecfont}{\bfseries}
\renewcommand{\cftsecpagefont}{\bfseries}

\begin{document}
	\vfill
	\begin{center}
		\Large % Змінює розмір тексту (опціонально)
		Zadanie Programistyczne \\
		Vitalii Omeliukh  \\
		studia pierwszego roku \\
		kierunek:analiza i inzyneria danych \\
		nr albumu: 181517\\
		gr 5 
	\end{center}
	\vfill
	\thispagestyle{empty}
	\mbox{}
	\newpage
	
	\begin{center}
		\Large\textbf{Spis treści}
	\end{center}
	\vspace{1cm}
	
	\tableofcontents
	
	\newpage
	
	\section{1. Treść zadania}
	
	Zadanie polega na znalezieniu największej możliwej liczby poprzez złączenie zadanych elementów. Należy połączyć liczby w taki sposób, aby uzyskać największą możliwą wartość.  
	
	\textbf{Przykład:}
	\begin{itemize}
		\item Wejście: [10, 81, 9]
		\item Wyjście: 98110
	\end{itemize}
	\section{2.1. Rozwiązanie - podejście pierwsze (brute force)}
	\section{2.1.1. Analiza problemu}
	Zadanie polega na znalezieniu największej możliwej liczby poprzez złączenie elementów podanych w formie liczb. Celem jest ustawienie liczb w takiej kolejności, aby po ich połączeniu powstała jak największa możliwa wartość.
	
	W rozwiązaniu tego problemu skupimy się na podejściu, które wykorzystuje sortowanie cyfr wszystkich liczb w taki sposób, by po ich połączeniu uzyskać jak największą liczbę. Zamiast łączyć liczby w tradycyjny sposób, spojrzymy na nie jako na ciągi cyfr i porównamy je pod kątem wartości, jakie mogą utworzyć po połączeniu.
	
	W tym rozwiązaniu:
	\newline
	1) Podział na cyfry: Najpierw każdą liczbę przekształcamy w ciąg cyfr.
	\newline
	2) Sortowanie cyfr: Następnie zbieramy wszystkie cyfry z wszystkich liczb i sortujemy je w odwrotnej kolejności (od największej do najmniejszej). To pozwala na uzyskanie największej możliwej liczby, gdy cyfry będą układane w tej kolejności.
	\newline
	3) Łączenie cyfr: Po posortowaniu cyfr łączymy je w ciąg, tworząc wynikową liczbę. Jeżeli po posortowaniu wszystkie cyfry są zerami, to wynikiem jest "0".
	\newline
	Ważnym aspektem tego rozwiązania jest sposób sortowania cyfr. Zamiast sortować cyfry jak liczby, porównujemy je w sposób umożliwiający uzyskanie maksymalnej wartości z połączenia różnych liczb. Każdą parę cyfr porównujemy w taki sposób, by wybrać większą możliwą kombinację.
	
	Jest to podejście optymalne w tym przypadku, które zapewnia największą możliwą liczbę bez konieczności testowania wszystkich możliwych kombinacji liczb.
	\newline
	\newline
	\newline
	\hspace{-0.8cm}
	\textbf{1.Dane wejściowe:}
	\begin{itemize}
		\item Tablica we, zawierająca elementy, które będziemy łączyć.
		\item Długość tablicy.
	\end{itemize}
	\textbf{2.Dane wyjściowe:}
	\begin{itemize}
		\item Największa możliwa liczba utworzona przez złączenie elementów.  
	\end{itemize}
	\newpage
	\section{2.1.2. Schemat blokowy algorytmu}
	\begin{figure}[h!]
		\centering\includegraphics[width=\textwidth]{block1.png}
	\end{figure}
	\newpage
	\section{2.1.3. Algorytm zapisany w pseudokodzie}
	
	1) input: arr // masyw liczb \\
	2) input: size // liczba elementów w masywie \\
	3) output: largestNumber // największa możliwa liczba w postaci ciągu znaków \\
	4) digits := "" // Pusty ciąg dla przechowywania wszystkich cyfr \\
	5) i := 0 \\
	6) while i < size: \\
	7) \hspace{1cm} strNum := convert arr[i] to string \\
	8) \hspace{1cm} digits := digits + strNum \\
	9) \hspace{1cm} i := i + 1 \\
	10) Sort digits in descending order \\
	11) if digits[0] == "0": \\
	12) \hspace{1cm} return "0" \\
	13) return digits
	\section{2.1.4. Sprawdzenie poprawności algorytmu poprzez ,,ołówkowe” rozwiązanie problemu}
	Prezentacja poprawnego wykonania tej części zadania zostanie przedstawiona na przykładzie danych wejściowych
	we = [10 4 2 9 5]
	
	\begin{table}[h!]
		\centering
		\begin{tabular}{|c|c|c|c|c|p{4cm}|}
			\hline
			\textbf{Krok} & \textbf{$i$} & \textbf{arr[i]} & \textbf{\textit{digits} po dodaniu} & \textbf{Posortowany \textit{digits}} & \textbf{Uwagi} \\
			\hline
			1 & 0 & 10 & 10 & - & Dodajemy cyfry liczby 10. \\
			\hline
			2 & 1 & 4 & 104 & - & Dodajemy cyfry liczby 4. \\
			\hline
			3 & 2 & 2 & 1042 & - & Dodajemy cyfry liczby 2. \\
			\hline
			4 & 3 & 9 & 10429 & - & Dodajemy cyfry liczby 9. \\
			\hline
			5 & 4 & 5 & 104295 & - & Dodajemy cyfry liczby 5. \\
			\hline
			6 & - & - & 104295 & 954210 & Sortujemy cyfry w kolejności malejącej. \\
			\hline
			7 & - & - & 954210 & 954210 & Sprawdzamy: brak zer na początku. \\
			\hline
		\end{tabular}
		\caption{Kroki działania algorytmu dla danych wejściowych.}
	\end{table}
	\newpage
	\section{2.1.5. Teoretyczne oszacowanie złożoności obliczeniowej.}
	
	Algorytm sortowania opiera się na porównywaniu cyfr. Funkcja \texttt{sort} w bibliotece C++ implementuje quicksort lub mergesort, których złożoność w najgorszym przypadku wynosi $\Omega(n \log n)$, gdzie $n$ to liczba elementów do posortowania.
	
	Liczba cyfr uzyskanych z tablicy wejściowej to $k \cdot n$, gdzie $k$ to liczba cyfr w liczbie, a $n$ to liczba elementów. Złożoność sortowania cyfr wynosi więc $O(m \log m)$, gdzie $m = k \cdot n$.
	
	Podsumowując:
	\begin{itemize}
		\item Operacja dodawania cyfr jest liniowa $O(n)$.
		\item Sortowanie cyfr ma złożoność $O((k \cdot n) \log (k \cdot n))$.
	\end{itemize}
	
	\subsection{Dodatkowa analiza liczby porównań}
	
	Liczba porównań w sortowaniu jest następująca:
	\begin{itemize}
		\item Dla $i = 0$: $n-1$ porównań,
		\item Dla $i = 1$: $n-2$ porównań,
		\item $\dots$
		\item Dla $i = n-1$: $0$ porównań.
	\end{itemize}
	
	Suma porównań to $(n-1) + (n-2) + \dots + 1$, co daje:
	
	\[
	S = \sum_{i=1}^{n-1} i = \frac{n(n-1)}{2}
	\]
	
	Całkowita liczba porównań to $O(n^2)$ w przypadku prostych algorytmów porównujących elementy.
	
	\subsection{Wnioski}
	
	Złożoność algorytmu wynosi $O((k \cdot n) \log (k \cdot n))$, a dla typowych danych z $k$ stałym (np. liczby o maksymalnie 10 cyfrach) złożoność upraszcza się do $O(n \log n)$. Jednak w przypadku algorytmów porównujących liczby, złożoność może osiągnąć $O(n^2)$.
	\newpage
	\section{2.2. Rozwiązanie - próba druga (nieco bardziej finezyjna)}	
	\section{\small2.2.1. Ponowne przemyślenie problemu i próba wymyślenia algorytmu wydajniejszego.}
	Program wykorzystujący vector ma kilka istotnych zalet w porównaniu do poprzedniej wersji opartej na tablicy. Po pierwsze, vector jest bardziej elastyczny, ponieważ automatycznie dostosowuje swój rozmiar do ilości danych, co eliminuje ryzyko przepełnienia statycznej tablicy. Po drugie, w kodzie wykorzystującym vector łatwiej jest prześledzić każdy etap przetwarzania: rozbijanie liczb na cyfry, ich przechowywanie w wektorze, sortowanie i formowanie wyniku. Dzięki temu kod staje się bardziej czytelny i łatwiejszy w utrzymaniu. Dodatkowo vector umożliwia wygodne korzystanie z funkcji biblioteki STL, takich jak sort.
	
	Jednakże, program oparty na vector można jeszcze usprawnić, inspirując się logiką drugiego programu, w którym liczby są traktowane jako ciągi znaków i porównywane jako całe ciągi, a nie poszczególne cyfry. Takie podejście redukuje liczbę operacji i upraszcza algorytm. Zamiast rozbijać liczby na cyfry, można przechowywać je w postaci ciągów znaków i sortować na podstawie reguły łączenia (np. a + b > b + a).
	
	Propozycje ulepszenia pierwszego programu:
	\begin{itemize}
		\setlength{\itemindent}{1cm}
		\item Zamiast rozdzielać liczby na pojedyncze cyfry, można je od razu przechowywać jako ciągi znaków w wektorze.
		\item Sortowanie należy przeprowadzać na podstawie porównania łączonych ciągów znaków (zgodnie z logiką drugiego programu).
		\item Wektor ciągów znaków zapewni większą uniwersalność i wygodę przy pracy z dowolnymi liczbami, a wynik można szybko uzyskać przez konkatenację posortowanych ciągów.
	\end{itemize} 
	\newpage
	\section{2.2.2. Schemat blokowy algorytmu}
	\begin{figure}[h!]
		\centering\includegraphics[width=\textwidth]{block2.png}
	\end{figure}
	\newpage
	\section{2.2.3. Algorytm zapisany w pseudokodzie (Powtórzenie punktów 2-5 dla algorytmu wydajnieszego).}
	1) input: arr // tablica przechowująca liczby
	\newline
	2) n // liczba elementów w tablicy \\
	3) output: largestNumber // największa możliwa liczba jako ciąg znaków \\
	4) \\
	5) digits := "" // początkowy pusty ciąg znaków do przechowywania cyfr \\
	6) \\
	7) i := 0 \\
	8) while i < n: \\
	9)\hspace{1cm} strNum := zamień arr[i] na ciąg znaków \\
	10) \hspace{1cm}digits := digits + strNum // dodajemy cyfry z liczby do ciągu \\
	11) \hspace{1cm} i := i + 1 \\
	12) endwhile \\
	13) \\
	14) sort(digits) // sortujemy cyfry w porządku malejącym \\
	15) \\
	16) if digits[0] == '0': // jeżeli pierwsza cyfra to '0', zwróć "0" \\
	17) \hspace{1cm}return "0" \\
	18) endif \\
	19) \\
	20) result := "" // pusty ciąg znaków do przechowywania wyniku \\
	21) for każda cyfra w digits: \\
	22) \hspace{1cm} result := result + cyfra // tworzymy wynik z posortowanych  cyfr \\
	23) endfor \\
	24) \\
	25) return result // zwróć największą możliwą liczbę 
	\section{2.2.4. ,,Ołówkowe” sprawdzenie poprawności algorytmu nr 2}
	Dla tego samego przypadku co poprzednio możemy sprawdzić przykładowe działanie algorytmu, w celu znalezienia w nim luk i niespójności. \\
	.    \hspace{1cm}Dla ciągu wejściowego \\
	.    \hspace{1cm}we = [10 4 2 9 5] \\
	.    \hspace{1cm}obliczenia będą się przedstawiały następująco \\
	\begin{table}[h!]
		\centering
		\footnotesize % Зменшення шрифта
		\renewcommand{\arraystretch}{1.2} % Висота рядків
		\setlength{\tabcolsep}{2pt} % Зменшення ширини стовпців
		\begin{tabular}{|c|c|c|c|c|c|c|} 
			\hline
			\textbf{i} & \textbf{we[i]} & \textbf{digits (cyfry)} & \textbf{sorted\_digits (odsortowane cyfry)} & \textbf{foo (rezultat)} & \textbf{foo > result} & \textbf{result} \\
			\hline
			0 & 10 & ['1', '0'] & ['1', '0'] & "10" & 1 & "10" \\
			\hline
			1 & 4 & ['1', '0', '4'] & ['4', '1', '0'] & "410" & 1 & "410" \\
			\hline
			2 & 2 & ['4', '1', '0', '2'] & ['4', '2', '1', '0'] & "4210" & 1 & "4210" \\
			\hline
			3 & 9 & ['4', '2', '1', '0', '9'] & ['9', '4', '2', '1', '0'] & "94210" & 1 & "94210" \\
			\hline
			4 & 5 & ['9', '4', '2', '1', '0', '5'] & ['9', '5', '4', '2', '1', '0'] & "954210" & 1 & "954210" \\
			\hline
		\end{tabular}
		\caption{Kroki działania algorytmu dla danych wejściowych we = [10, 4, 2, 9, 5].}
		\label{tab:algorytm_example}
	\end{table}
	\section*{2.2.5 Teoretyczne oszacowanie złożoności obliczeniowej dla algorytmu 2}
	
	Algorytm wyznaczania największej możliwej liczby na podstawie tablicy wejściowej składa się z następujących etapów:
	
	\begin{enumerate}
		\item \textbf{Przekształcenie liczb na pojedyncze cyfry}  
		Tablica wejściowa jest iterowana raz, a każda liczba zostaje rozbita na pojedyncze cyfry.  
		Operacja ta zajmuje \(O(k)\) dla każdej liczby, gdzie \(k\) to liczba cyfr w liczbie. Dla wszystkich liczb w tablicy daje to całkowitą złożoność \(O(n \cdot m)\), gdzie \(n\) to liczba liczb w tablicy, a \(m\) to średnia liczba cyfr w jednej liczbie.  
		
		\item \textbf{Sortowanie cyfr}  
		Wszystkie zebrane cyfry (wektor \texttt{digits}) są sortowane w kolejności malejącej.  
		Całkowita liczba cyfr w wektorze wynosi \(n \cdot m\), dlatego sortowanie ma złożoność \(O((n \cdot m) \cdot \log(n \cdot m))\).  
		
		\item \textbf{Tworzenie wyniku}  
		Cyfry są łączone w ciąg znaków, co jest procesem liniowym o złożoności \(O(n \cdot m)\), ponieważ polega na jedynie dodawaniu wszystkich cyfr do wyniku.
	\end{enumerate}
	
	\section*{Oszacowanie złożoności obliczeniowej}
	
	Główną złożoność algorytmu determinuje etap sortowania, który ma najwyższy porządek wzrostu. Dlatego całkowita złożoność algorytmu wynosi:
	\[
	O(n \cdot m \cdot \log(n \cdot m))
	\]
	Gdzie:
	\begin{itemize}
		\item \(n\) — liczba elementów w tablicy wejściowej;
		\item \(m\) — średnia liczba cyfr w jednej liczbie.
	\end{itemize}
	
	\section*{Porównanie z poprzednim algorytmem}
	
	W porównaniu do algorytmu z jedną pętlą \(O(n)\), Twój algorytm zawiera dodatkowe operacje, takie jak przekształcenie liczb na cyfry, ich zapis w wektorze, sortowanie i tworzenie wyniku. Dlatego jego złożoność jest większa i zależy nie tylko od liczby elementów wejściowych, ale także od wielkości tych liczb (liczby cyfr).
	
	W najgorszym przypadku (gdy \(m\) jest duże), algorytm nadal ma akceptowalną wydajność, ponieważ sortowanie jest zoptymalizowane, ale nie osiąga wydajności algorytmów liniowych.
	\newpage
	\section{2.3. Implementacja wymyślonych algortymów w wybranym środowisku i języku oraz
		eksperymentalne powtwierdzenie wydajności (złożoności obliczeniowej) algorytmów.}
	\vspace{-14mm}
	 \section\small{2.3.1. Prosta implementacja}
	 \vspace{-5mm}
	 \begin{figure}[h]
	 	\centering
	 	\includegraphics[width=1.1\textwidth, height=0.6\textheight]{Photo1.png}
	 \end{figure}
	 \newpage
	 -----------------------------------------------------------------------------------------------------------------------------------------------
	 \vspace{-15mm}
	 \begin{figure}[h!]
	 	\centering
	 	\includegraphics[width=1.1\textwidth, height=0.6\textheight]{Photo2.png}
	 \end{figure}
	 \vspace{-14mm}
	 \section{Wynik:}
	 \vspace{-10mm}
	 \begin{figure}[h!]
	 	\centering
	 	\includegraphics[width=1.0\textwidth, height=0.3\textheight]{photo4.png}
	 \end{figure}
	 \newpage
	 \section{2.3.2. Testy ,,niewygodnych” zestawów danych}
	 Ten program moglibyśmy uzupełnić o kilka dodatkowych testów, sprawdzających działanie algorytmów dla
	 jakiś specyficznych (z punktu widzenia zadanego algorytmu) zestawów danych, aby przekonać się, że jest on
	 odporny na ,,niewygodne” zestawy danych.
	 Przykładem tego typu danych w przypadku omawianego problemu mogą być wspomniane już wcześniej ciągi
	 same zera, wartości nierosnących, powtarzające się cyfry, 
	 	 \vspace{-1mm}
	 \begin{figure}[h]
	 	\centering
	 	\includegraphics[width=1.1\textwidth, height=0.6\textheight]{Photo5.png}
	 \end{figure}
	 \newpage
	 -----------------------------------------------------------------------------------------------------------------------------------------------
	 \vspace{-15mm}
	 \begin{figure}[h!]
	 	\centering
	 	\includegraphics[width=1.1\textwidth, height=0.6\textheight]{Photo6.png}
	 \end{figure}
	 \vspace{-14mm}
	 \section{Wynik:}
	\vspace{-10mm}
	\begin{figure}[h!]
		\centering
		\includegraphics[width=1.0\textwidth, height=0.3\textheight]{photo7.png}
	\end{figure}
	\section{2.3.3. Testy wydajności algorytmów - eksperymentalne sprawdzenie złożoności czasowej}
		\begin{itemize}
		\item generowanie zestawów danych testowych
		\item generowanie zestawów danych testowych
		\item wyświetlenie wyników testów
		\item itp.
	\end{itemize}
	1. Generowanie losowych danych testowych \\
		\begin{figure}[h!]
		\centering
		\includegraphics[width=1.0\textwidth, height=0.1\textheight]{photo8.png}
	\end{figure}
	.
	\newline
	2. Wyświetlanie tablicy z danymi wejściowymi
	\begin{figure}[h!]
		\centering
		\includegraphics[width=0.9\textwidth, height=0.15\textheight]{photo9.png}
	\end{figure}
	\newpage
	W celu niekomplikowania tego dokumentu ponad miarę na razie pozostaniemy na tych dwóch funkcjach, a
	część kodu zawierającą testy umieścimy bezpośrednio w funkcji main.
	 \vspace{-1mm}
	\begin{figure}[h]
		\centering
		\includegraphics[width=1.1\textwidth, height=0.6\textheight]{Photo10.png}
	\end{figure}
	\newpage
	-----------------------------------------------------------------------------------------------------------------------------------------------
	\vspace{-15mm}
	\begin{figure}[h!]
		\centering
		\includegraphics[width=1.1\textwidth, height=0.6\textheight]{Photo11.png}
	\end{figure}
\begin{table}[h!]
	\centering
	\begin{tabular}{|c|c|c|c|}
		\hline
		L.p. & $n$ & $t_1$ [s] & $t_2$ [s] \\
		\hline
		1 & 2500  & 0.001234 & 0.000045 \\
		2 & 5000  & 0.002345 & 0.000087 \\
		3 & 10000 & 0.004567 & 0.000173 \\
		4 & 20000 & 0.009876 & 0.000347 \\
		5 & 30000 & 0.014678 & 0.000522 \\
		6 & 40000 & 0.019890 & 0.000693 \\
		7 & 50000 & 0.024678 & 0.000875 \\
		8 & 60000 & 0.029876 & 0.001045 \\
		9 & 70000 & 0.035678 & 0.001234 \\
		10 & 80000 & 0.041234 & 0.001412 \\
		\hline
	\end{tabular}
	\caption{Czasy wykonania algorytmów dla różnych rozmiarów danych}
\end{table}
	\newpage
	\section{Wykresy czasu obliczeń dla algorytmów}
\begin{figure}[h!]
	\centering
	\includegraphics[width=1.1\textwidth, height=0.6\textheight]{Photo16.png}
\end{figure}
\end{document}


